
%%%%%%%%%%%%%%%%%%%%%%%%%%%%%%%%%%%%%%%%%%%%%%%%%%%%%%%%%%%%%%%%
% %                                                            %         
% Isaias Munoz Venegas %                                       %
% ECE 351 %                                                    %
% Lab 8 %                                                       %
% 10/12/22 %                                                    %
% Any other necessary information needed to navigate the file  %
% %                                                            %
%%%%%%%%%%%%%%%%%%%%%%%%%%%%%%%%%%%%%%%%%%%%%%%%%%%%%%%%%%%%%%%%

\documentclass[12pt,a4paper]{article}
\usepackage[utf8]{inputenc}
\usepackage[greek,english]{babel}
\usepackage{alphabeta} 
\usepackage[pdftex]{graphicx}
\usepackage[top=1in, bottom=1in, left=1in, right=1in]{geometry}
\linespread{1.06}
\setlength{\parskip}{8pt plus2pt minus2pt}
\widowpenalty 10000
\clubpenalty 10000
\newcommand{\eat}[1]{}
\newcommand{\HRule}{\rule{\linewidth}{0.5mm}}
\usepackage[official]{eurosym}
\usepackage{enumitem}
\setlist{nolistsep,noitemsep}
\usepackage[hidelinks]{hyperref}
\usepackage{cite}
\usepackage{lipsum}
\begin{document}
%===========================================================
\begin{titlepage}
\begin{center}
% Top 
\includegraphics[width=0.55\textwidth]{cut-logo-en}~\\[2cm]
% Title
\HRule \\[0.4cm]
{ \LARGE 
  \textbf{Project Report for ECE 351}\\[0.4cm]
  \emph{Lab 7: Fourier Series Approximation of a Square Wave}\\[0.4cm]
}
\HRule \\[1.5cm]
% Author
{ \large
  Isaias Munoz  \\[0.1cm]
  \today\\[0.1cm]
  %#\texttt{user@cut.ac.cy}
}
\vfill
\textsc{\Large University of Idaho}\\
\\
% Bottom
%{\large \today}
 
\end{center}
\end{titlepage}
%\begin{abstract}
%\lipsum[1-2]
%\addtocontents{toc}{\protect\thispagestyle{empty}}
%\end{abstract}
\newpage
%===========================================================
\tableofcontents
\addtocontents{toc}{\protect\thispagestyle{empty}}
\newpage
\setcounter{page}{1}
%===========================================================
%===========================================================
\section{Introduction}\label{sec:intro}

The purpose of this lab is to use Fourier Series to approximate periodic time-domain signals.

%\section{Equations}\label{sec:lit-rev}




\section{Methodology}\label{sec:meth}
\subsection{Part 1}
The first prelab task was to solve the Fourier series of Figure 1 using these equations. 

\[x(t) = \frac{1}{2}a_0+\sum_{n=1}^{\infty}a_kcos(w_0t)+b_ksin(kw_0t)\]
\[a_k = \frac{2}{T}\int_{0}^{T}x(t)cos(kw_0t)dt\]
\[b_k = \frac{2}{T}\int_{0}^{T}x(t)sin(kw_0t)dt\]
\[w_0 = \frac{2\pi}{T}\]
Realizing since this is an odd function  $a_k$ and $a_0$ is zero and only $b_k$ needs solved. 
\[a_k = 0\]
\[b_k = \frac{2}{k\pi}[1-cos(k\pi)]\]
\[x(t) = \sum_{n=1}^{\infty}b_ksin(kw_0t)\]

 
 
 
 
 
\begin{figure}[h]
\begin{subfigure}{ 1\textwidth}
\includegraphics[width=.9\linewidth, height=5
cm]{fser.png}
\end{subfigure}
\caption{Block Diagram}
\label{fig:image2}
\end{figure}
\clearpage

The first task was to solve for $a_k$ and $b_k$ explicitly and obtain values. Figure 2 shows the code for doing so.
 
\begin{figure}[h]
\begin{subfigure}{ 1\textwidth}
\includegraphics[width=.9\linewidth, height=16
cm]{tk1.png}
\end{subfigure}
\caption{Block Diagram}
\label{fig:image2}
\end{figure}

I made a function to print out the values of $b_k$ at k=  1,2 and 3. This was straightforward and the challenging part was whether or not $b_k$ was found correctly in the pre-lab. The confirmation of the values can be found under the result section. 

I then moved to task 2 and it was to plot the Fourier of Figure 1 at different k values. The code is in Figure 3 below. This was tricky because I ran into a few problems doing so.
\clearpage
\begin{figure}[h]
\begin{subfigure}{ 1\textwidth}
\includegraphics[width=.9\linewidth, height=16
cm]{forier.png}
\end{subfigure}
\caption{Block Diagram}
\label{fig:image2}
\end{figure}

As can be seen I ultimately left everything in variables for $w$ and $T$. I then wrote the for loop to encapsulate both the $b_k$ term with the Fourier and obtain that expression. I then called the function and passed it the time x-axis and the number of iterations I wanted it to add. The issue came with line 66 and having $range(1,k+1)$ instead of $range(1,k)$ because this produces and error and your not starting at them same place. Aside from that it worked fine. The plots for different $k$ values are found in the result section.


\clearpage

\section{Results}\label{sec:res}

\subsection{Part 1}




\begin{figure}[h]
\centering
\begin{subfigure}{ 1\textwidth}
\includegraphics[width=.6\linewidth, height=3
cm]{ver.png}
\end{subfigure}
\caption{ Verifying the $b_k$ and $a_k$ terms  }
\label{fig2:image22}
\end{figure}


\begin{figure}[h]
\centering
\begin{subfigure}{ 1\textwidth}
\includegraphics[width=.9\linewidth, height=10
cm]{plo1.png}
\end{subfigure}
\caption{ Results of $k=1,3,15$ }
\label{fig2:image22}
\end{figure}

\begin{figure}[h]
\centering
\begin{subfigure}{ 1\textwidth}
\includegraphics[width=.9\linewidth, height=10
cm]{plo2.png}
\end{subfigure}
\caption{ Results of $k=50,15,1500$ }
\label{fig2:image22}
\end{figure}
\clearpage






\section{Questions}\label{sec:res}

1. Is x(t) an even or an odd function? Explain why. \newline

\noindent  It is an odd function because it looks like a sine and sine is an odd function.

\noindent 2. Based on your results from Task 1, what do you expect the values of a2, a3, . . . , an to be?
Why? \newline 
\noindent 0 because its on going forever and it's always going to be an odd function.

\noindent 3. How does the approximation of the square wave change as the value of N increases? In what
way does the Fourier series struggle to approximate the square wave?\newline
\noindent The more iterations the close it looks like the square wave. The value evaluated for example inputting an x term or number of iterations into the sine and cosine outputs a y value that the more the iterations are the closer they are hence the flatter the top of the square wave it is.

\noindent 4. What is occurring mathematically in the Fourier series summation as the value of N increases?. \newline
\noindent The frequency increases meaning less variation between the rising and decreasing of the wave or a flatter top meaning a good perfect square approximation.

\noindent 5. Leave any feedback on the clarity of lab tasks, expectations, and deliverables. \newline
It was straightforward I think.




%\lipsum[7-8]\cite{knuthwebsite}
%===========================================================
%===========================================================
\bibliographystyle{ieeetr}
\bibliography{refs}
\end{document}